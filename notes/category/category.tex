\documentclass{article}

\usepackage{amsthm}
\usepackage{amssymb}
\usepackage{enumerate}
\usepackage{diagrams}

\newtheorem*{cly}{Corollary}
\newtheorem*{define}{Definition}
\newtheorem*{eg}{Example}
\newtheorem*{prf}{Proof}
\newtheorem*{rem}{Remark}
\newtheorem*{thm}{Theorem}


\newcommand{\abs}[1]{\left| {#1} \right|}

\newcommand{\Hom}{\mathop{\mathrm{Hom}}}
\newcommand{\id}{\mathop{\mathrm{id}}}
\newcommand{\dom}{\mathop{\mathrm{dom}}}
\newcommand{\codom}{\mathop{\mathrm{codom}}}
\newcommand{\obj}{\mathop{\mathrm{Obj}}}
\newcommand{\arrow}{\mathop{\mathrm{Arrow}}}

\title{Notes for Categoty Theory}
\begin{document}
\maketitle

\section{Basic Concepts}
\subsection{Category}
In traditional mathematics based on set,
we study set and maps(also called functions) between sets.
A set is a collction of elements, and a map transform an element to another.

However, in category theory, the basic concepts are class, object and morphism.
A class is the analogy to set and an object is the analogy to element,
therefore morphism is the analog to mapping between sets.

\begin{define}[category]
A category $\mathcal{C}$ consists of a class of objects, and for any
two objects $A, B$, there are disjoint sets of morphisms
$\Hom(A, B)$ between them, for three objects $A, B$ and $C$,
there is a map $\Hom(A, B) \times \Hom(B, C) \to \Hom(A, C)$,
satisfy:
\begin{enumerate}[1).~]
\item for any object $A$, exists $\id_{A} \in \Hom(A, A)$,
such that, $\id_{A} \circ f = f$, $g \circ \id_{A} = g$;
\item for any $f \in \Hom(A, B)$, $g \in \Hom(B, C)$,
$h \in \Hom(C, D)$, $(f \circ g) \circ h = f \circ (g \circ h)$.
\end{enumerate}
\end{define}

\begin{rem}
If $(f,g) \to h$, we often write $h = g \circ f$, notice the
order. Object may not be set and morphism may not be map
between sets.
\end{rem}

\begin{eg}
Sets and functions between them form a catgory, the category
of sets.
\end{eg}

\begin{rem}
Category can be represenet by a multi-graph. The objects in
the category are node of the graph, the morphisms are the
directed edges(we call arrow here). And for each arrow, it
has a source(domain) node and a target(codomain) node.
Of course the source and target of an arrow can be the same,
we then call that arrow an endomorphism.
\end{rem}

Let $A, B, C$ be objects of category $\mathcal C$,
and $f \in \hom(A, B)$, $g \in \hom(B, C)$, $h \in \hom(A, C)$
we say the following diagram commutes if $g \circ f = h$.
\begin{diagram}
A & \rTo^{f}  & B \\
  & \rdTo_{h} & \dTo>{g} \\
  &           & C
\end{diagram}
Generally, the diargam can have more vertices and edges, and is commutative as
long as all path connecting two vertics are identical morphisms.

\subsection{Functor and Cofunctor}
Functors are map between categories. Remember category
consists of objects and morphisms, so a functor must map
object to object and map morphism to morphism.

\begin{define}[functor]
Let $\mathcal C$ and $\mathcal D$ be two categories,
a functor $\mathcal F$ assigned each object $A \in \mathcal C$
an object $\mathcal F(A) \in \mathcal D$;
and for each morphism $f \in \Hom(A, B)$,
$\mathcal F(f) \in \Hom(\mathcal F(A), \mathcal F(B))$,
such that $\mathcal F(\id_A) = \id_{\mathcal F(A)}$,
and $\mathcal F(g \circ f) = \mathcal F(g) \circ \mathcal F(f)$.
\end{define}

A functor maps a commutative diagram in $\mathcal C$ to one in $\mathcal D$.

\begin{minipage}{0.3\textwidth}
\begin{diagram}
A & \rTo^{f}          & B \\
  & \rdTo_{g \circ f} & \dTo>{g} \\
  &                   & C
\end{diagram}
\end{minipage}
\begin{minipage}{0.3\textwidth}
\begin{diagram}
 &  & \\
 & \rTo^{\mathcal F} & \\
 &  &
\end{diagram}
\end{minipage}
\begin{minipage}{0.3\textwidth}
\begin{diagram}
\mathcal F(A) & \rTo^{\mathcal F(f)}          & \mathcal F(B) \\
              & \rdTo_{\mathcal F(g \circ f)} & \dTo>{\mathcal F(g)} \\
              &                               & \mathcal F(C)
\end{diagram}
\end{minipage}

\begin{define}[cofunctor]
A cofunctor, or contravariant functor is almost like a functor,
except for $\mathcal F(f) \in \Hom(B, A)$ and
$\mathcal F(g \circ f) = \mathcal F(f) \circ \mathcal F(g)$.
\end{define}

A cofunctor maps a commutative diagram in $\mathcal C$ to one in $\mathcal D$
and reverse all morphisms.

\begin{minipage}{0.3\textwidth}
\begin{diagram}
A & \rTo^{f}          & B \\
  & \rdTo_{g \circ f} & \dTo>{g} \\
  &                   & C
\end{diagram}
\end{minipage}
\begin{minipage}{0.3\textwidth}
\begin{diagram}
 & & \\
 & \rTo^{\mathcal F} & \\
 & &
\end{diagram}
\end{minipage}
\begin{minipage}{0.3\textwidth}
\begin{diagram}
\mathcal F(A) & \lTo^{\mathcal F(f)}          & \mathcal F(B) \\
              & \luTo_{\mathcal F(g \circ f)} & \uTo>{\mathcal F(g)} \\
              &                               & \mathcal F(C)
\end{diagram}
\end{minipage}

\begin{eg}
For any category $\mathcal C$ whose objects has an underlying set,
there is a functor from $\mathcal C$ to the category of set,
which simply sends an object to its underlying set,
and sends the morphisms to map between sets.
This functor is called the forgetful functor.
\end{eg}


\subsection{Natural Transformation}
Natural Transformations are morphisms between functors.
\begin{define}[natural transformation]
Let $\mathcal F, \mathcal G$ be two functors, both from $\mathcal C$ to $\mathcal D$.
A natural transformation $\sigma$ sends each object $A$ of $\mathcal C$ to
a morphism in $\hom_{\mathcal D}(\mathcal F(A), \mathcal G(A))$.
such that for any object $B$ in $\mathcal C$ and morphism $f$ in $\hom_{\mathcal C}(A, B)$,
we have
$$\sigma(B) \circ F(f) = G(f) \circ \sigma(A)$$
\end{define}
That is the following diagram commutes.
% A --f-> B
% F(A) --F(f)-> F(B)
%  |              |
% s(A)          s(B)
%  |               |
% G(A) --G(f)-> G(B)
\begin{diagram}
\mathcal F(A)    & \rTo^{\mathcal F(f)} & \mathcal F(B) \\
\dTo<{\sigma(A)} &                      & \dTo>{\sigma(B)} \\
\mathcal G(A)    & \rTo_{\mathcal G(f)} & \mathcal G(B)
\end{diagram}

\section{Constructions}
\subsection{Functor Category}
Let $\mathcal F, \mathcal G, \mathcal H$ be functors from $\mathcal C$ to $\mathcal D$,
and $\sigma(A) : \mathcal F(A) \to \mathcal G(A)$, $\tau : \mathcal G(A) \to \mathcal H(A)$.
Define $(\tau \circ \sigma)(A) : \mathcal F(A) \to \mathcal H(A)$ by $\tau(A) \circ \sigma(A)$.
Then $\tau \circ \sigma$ is a natural transformation from $\mathcal F$ to $\mathcal H$.
As shown from the commutative diagram below:
\begin{diagram}
\mathcal F(A)    & \rTo^{\mathcal F(f)} & \mathcal F(B) \\
\dTo<{\sigma(A)} &                      & \dTo>{\sigma(B)} \\
\mathcal G(A)    & \rTo_{\mathcal G(f)} & \mathcal G(B) \\
\dTo<{\tau(A)} &                      & \dTo>{\tau(B)} \\
\mathcal H(A)    & \rTo_{\mathcal H(f)} & \mathcal H(B)
\end{diagram}

Let $\mathcal F : \mathcal C \to \mathcal D$ be the objects and $\circ$ be the morphisms,
this category is called the functor category.

\subsection{Category of Categories}


\section{Operations}
\subsection{Product and Coproduct}
Let $\mathcal{C}$ be a category, $A$ and $B$ be objects of
$\mathcal{C}$. When $A$ and $B$ are sets, we can construct
the cartesian product and disjoint union of them, now we
define the analogy terms in the language of category.

\begin{define}[product]
The product of $A$ and $B$ is a object $C$, with morphisms
$p: C \to A$ and $q: C \to B$, such that for any object
$X$ in $\mathcal{C}$, and morphisms $f: X \to A$, $g: X \to B$,
exists unique morphism $\theta : X \to C$,
$f = p \circ \theta$, $g = q \circ \theta$.
\end{define}
The product can be expressed by the following comutative diagram
\begin{diagram}
    &          & C            &          & \\
    & \ldTo{p} & \uTo{\theta} & \rdTo{q} & \\
  A & \lTo{f}  & X            & \rTo{g}  & B
\end{diagram}
When $A$, $B$ are sets, $C = A \times B$, $p$ and $q$ are
projection maps, $\theta(x)$ can be constructed by
$\theta(x) = (f(x), g(x))$.

\begin{define}[coproduct]
The coproduct of $A$ and $B$ is a object $D$, with morphisms
$\alpha : A \to D$, $\beta : B \to D$, such that for any
object $X$ in $\mathcal D$, and morphisms $f: A \to X$,
$g: B \to X$, exists unique morphism $\theta : D \to X$,
$f = \theta \circ \alpha$, $g = \theta \circ \beta$.
\end{define}
The coproduct can be expressed by the following commutative diagram
\begin{diagram}
  A & \rTo{f}       & X            & \lTo{g}      & B \\
    & \rdTo{\alpha} & \uTo{\theta} & \ldTo{\beta} & \\
    &               & D            &              &
\end{diagram}
When $A$ and $B$ are sets, $D = A \coprod B$, the disjoint union of $A$ and $B$.
$\alpha$ and $\beta$ are inclusion maps. $\theta(c)$ can be constructed
by $\theta(a') = f(a)$ and $\theta(b') = g(b)$.
Coproduct also called sum in some books.

Both the product and coproduct can be defined for a familly of objects $\{A_i\}_{i \in I}$.
Where the product is denoted by $\prod_{i \in I} A_i$ and the coproduct is denoted by $\coprod_{i \in I} A_i$.


\subsection{Pushout and Pullback}

\subsection{Limit and Colimit}

\section{Special Objects}
\subsection{Initial Object and Final Object}
In a category, if for any object $O$, there is exactly one
morphism from $X$ to $A$, then $X$ is called an initial object.
If for any object $A$, there is exactly one morphism from $A$ to $Y$,
then $Y$ is called a final object.


\section{Functional Category Theory}
Let $A, B, \dots$ be types, and $f, g, \dots$ be transformations between types.
For each type $X$, we have the identity transformation $\id_X$.
For $f : A \to B$ and $g : B \to C$, we have $g \circ f : A \to C$.

Then the types and transformations form a category.
Types are also called objects and transformations are also called arrows or morphisms.

For arrow $f : A \to B$, we say $f$ has type $A \to B$,
and $A$ is the domain of $f$, $B$ is the codomain of $f$.
When an object $A$ is a also a set, we say $a$ has type $A$ is $a \in A$.

Let $\mathfrak{C}, \mathfrak{D}$ be two categories, a functor $\mathcal{F}$
maps an arrow of $\mathfrak{C}$ to an arrow of $\mathfrak{D}$
such that $\mathcal{F}(g \circ f) = \mathcal{F}(g) \circ \mathcal{F}(f)$.
It's easy to verify that $\mathcal{F}(\id_X)$ is also an identity arrow.
So for an object $X \in \mathfrak{C}$, $\mathcal{F}$ will map $X$ to
$\dom \mathcal{F}(\id_X) = \codom \mathcal{F}(\id_X)$.

Locally, a functor has type $(A \to B) \to (C \to D)$.

Let $\mathcal{F}, \mathcal{G} : \mathfrak{C} \to \mathfrak{D}$ be functors,
$\tau : \obj(\mathfrak{C}) \to \arrow(\mathfrak{D})$ is a morphism of functors if
for any $f : A \to B$ in $\mathfrak{C}$, $\tau(B) \circ \mathcal{F}(f) = \mathcal{G}(f) \circ \tau(A)$.
Or equivalently, $\tau(\codom f) \circ \mathcal{F}(f) = \mathcal{G}(f) \circ \tau(\dom f)$

\subsection*{Exercise}
\begin{enumerate}[1.]
\item Prove that $\mathcal{F}(\id_X)$ is also an identity arrow.
\end{enumerate}

\subsection{Functor}
A functor may be viewed an overloaded function on both $\obj(\mathfrak C)$ and
$\arrow(\mathfrak C)$.

Let $\mathcal F : \mathfrak A \to \mathfrak B$ be a functor,
for $f : x \to y \in \hom(x, y), x, y \in \mathfrak A$,
$\mathcal F(f)$ is an arrow in $\mathfrak B$.


\subsection{Natural Transformation}
A natural transformation
$\sigma : \mathcal S \to \mathcal T,
\mathcal S, \mathcal T : \mathfrak A \to \mathfrak B$ has type
$\obj(A) \to \hom(B)$.


\subsection{Function Category}

\subsection{Category of Categories}

\subsection{Monad}

\subsection{Abelian Category}
For each $X, Y \in \mathcal{C}$, $\hom(X, Y)$ is an abelian group,
that is, there is a operation $\cdot : \hom(X, Y) \times \hom(X, Y) \to \hom(X, Y)$,
the morphism $\circ : \hom(X, Y) \times \hom(Y, Z) \to \hom(X, Z)$ is bilinear.
That is $(f_1 \cdot f_2) \circ g = (f_1 \cdot g) \circ (f_2 \cdot g)$
and $f \circ (g_1 \cdot g_2) = (f \cdot g_1) \circ (f \cdot g_2)$.


\section{Algebra and Coalgebra}
\subsection{Review of $R$-Algebra}
Let $R$ be a ring with identity, an $R$-algebra is an $R$-module $A$ with
multiplication $\cdot : A \times A \to A$, and the multiplication
satisifies:
\begin{enumerate}
\item bilinear: $(r a_1) \cdot a_2 = r (a_1 \cdot a_2) = a_1 \cdot (r a_2)$
\item distribution law: $a_1 \cdot (a_2 + a_3) = a_1 \cdot a_2 + a_1 \cdot a_3$,
  $(a_1 + a_2) \cdot a_3 = a_1 \cdot a_3 + a_2 \cdot a_3$
\item associative law: $(a_1 \cdot a_2) \cdot a_3 = a_1 \cdot (a_2 \cdot a_3)$
\end{enumerate}
Therefore $(A, +, \cdot)$ is a ring.
We often omit $\cdot$ and write $a_1a_2$ instead of $a_1 \cdot a_2$.

Define a map $\sigma: R \to A$ by $r \to r 1_A$,
then $\sigma$ is a ring homomorphism from $R$ to $A$ and
$\sigma(R) \subseteq Z(A)$.

\subsection{Monad}
Let $\mathcal X$ be a category, a monad over $\mathcal X$ is a tuple $(\mathcal T, \mu, \eta)$,
where $\mathcal T$ is an endofunctor of $\mathcal X$, $\mu : \mathcal T^2 \to \mathcal T$
and $\eta : \mathcal I \to \mathcal T$ are natural transformations,
where $\mathcal I$ is the identity functor of $\mathcal X$.
$\mu$ and $\eta$ satisify certain commutative diagrams mentioned following.

\subsubsection{The Composition}
Let $\mathcal X$ be a category, and $\mathcal T$ an endofunctor of $\mathcal X$.
$\mu$ is a matural transform from $\mathcal T$ to $\mathcal T^2$.
That is, for $x, y \in \mathcal X$ and $f \in \hom_{\mathcal X}(x, y)$
we have the following commutative diagram in $\mathcal X$:
\begin{diagram}
\mathcal T^2(x) & \rTo^{\mathcal T^2(f)} & \mathcal T^2(y) \\
\dTo<{\mu(x)}   &                        & \dTo>{\mu(y)} \\
\mathcal T(x)   & \rTo^{\mathcal T(f)}   & \mathcal T(y)
\end{diagram}
If We apply $\mathcal T$ to this diagram then we got
\begin{diagram}
\mathcal T^3(x) & \rTo^{\mathcal T^3(f)} & \mathcal T^3(y) \\
\dTo<{\mathcal T(\mu(x))} &              & \dTo>{\mathcal T(\mu(y))} \\
\mathcal T^2(x) & \rTo^{\mathcal T^2(f)} & \mathcal T^2(y)
\end{diagram}
And if we apply $\mathcal T$ to $f : x \to y$,
that is replace $x, y, f$ by $\mathcal T(x)$,
$\mathcal T(y)$ and $\mathcal T(f)$ respectly,
then we got
\begin{diagram}
\mathcal T^3(x) & \rTo^{\mathcal T^3(f)} & \mathcal T^3(y) \\
\dTo<{\mu(\mathcal T(x))} &              & \dTo>{\mu(\mathcal T(y))} \\
\mathcal T^2(x) & \rTo^{\mathcal T^2(f)} & \mathcal T^2(y)
\end{diagram}

The latter two commutative diagrams can be connect to the first one

\begin{minipage}{.5\textwidth}
\begin{diagram}
\mathcal T^3(x) & \rTo^{\mathcal T^3(f)} & \mathcal T^3(y) \\
\dTo<{\mathcal T(\mu(x))} &              & \dTo>{\mathcal T(\mu(y))} \\
\mathcal T^2(x) & \rTo^{\mathcal T^2(f)} & \mathcal T^2(y) \\
\dTo<{\mu(x)}   &                        & \dTo>{\mu(y)} \\
\mathcal T(x)   & \rTo^{\mathcal T(f)}   & \mathcal T(y)
\end{diagram}
\\
\end{minipage}
\begin{minipage}{.5\textwidth}
\begin{diagram}
\mathcal T^3(x) & \rTo^{\mathcal T^3(f)} & \mathcal T^3(y) \\
\dTo<{\mu(\mathcal T(x))} &              & \dTo>{\mu(\mathcal T(y))} \\
\mathcal T^2(x) & \rTo^{\mathcal T^2(f)} & \mathcal T^2(y) \\
\dTo<{\mu(x)}   &                        & \dTo>{\mu(y)} \\
\mathcal T(x)   & \rTo^{\mathcal T(f)}   & \mathcal T(y)
\end{diagram}
\\
\end{minipage}
To construct a $\mathcal T$-algebra, we add a new condition
which requires the following diagram commutes for each $x \in \mathcal X$
\begin{diagram}
\mathcal T^3(x) & \rTo^{\mathcal T(\mu(x))} & \mathcal T^2(x) \\
\dTo<{\mu(\mathcal T(x))} & & \dTo>{\mu(x)} \\
\mathcal T^2(x) & \rTo^{\mu(x)} & \mathcal T(x)
\end{diagram}

\subsubsection{The Unit}
Let $\mathcal I$ be the identity functor of $\mathcal X$,
and $\eta$ a naturan transformation from $\mathcal I$ to $\mathcal T$:
\begin{diagram}
\mathcal I(x) & \rTo^{\mathcal I(f)} & \mathcal I(y) \\
\dTo<{\eta(x)} & & \dTo>{\eta(y)} \\
\mathcal T(x) & \rTo^{\mathcal T(f)} & \mathcal T(y)
\end{diagram}
Apply $\mathcal T$ to this commutative diagram
\begin{diagram}
\mathcal T(x) & \rTo^{\mathcal T(f)} & \mathcal T(y) \\
\dTo<{\mathcal T(\eta(x))} & & \dTo>{\mathcal T(\eta(y))} \\
\mathcal T^2(x) & \rTo^{\mathcal T^2(f)} & \mathcal T^2(y)
\end{diagram}
Apply $\mathcal T$ to $f : x \to y$:
\begin{diagram}
\mathcal T(x) & \rTo^{\mathcal T(f)} & \mathcal T(y) \\
\dTo<{\eta(\mathcal T(x))} & & \dTo>{\eta(\mathcal T(y))} \\
\mathcal T^2(x) & \rTo^{\mathcal T^2(f)} & \mathcal T^2(y)
\end{diagram}

And we also require that the following diagrams commutes
\begin{diagram}
\mathcal I(x) & \rTo^{\eta(x)} & \mathcal T(x) \\
\dTo<{\eta(x)} & & \dTo>{\mathcal T(\eta(x))} \\
\mathcal T(x) & \rTo_{\eta(\mathcal T(x))} & \mathcal T^2(x))
\end{diagram}

\subsection{$\mathcal T$-Algebra}
For a monad $(\mathcal T, \mu, \eta)$, the $\mathcal T$-algebra is a
pair $(x, h)$ where $x$ is an object of $\mathcal X$ and $h : \mathcal T(x) \to x$.
$h$ is called the structure map.

\end{document}
