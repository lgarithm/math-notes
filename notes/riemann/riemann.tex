% \documentclass[14pt]{extarticle}
\documentclass{article}
\usepackage{amssymb}
% \usepackage[T1]{fontenc}

\newcommand{\alt}{\mathop{\mathrm{Alt}}}

\title{Notes on Riemann Grometry}
\author{Lgarithm\\\mbox{lgarithm@gmail.com}}

\begin{document}
% \maketitle

\section{Multilinear Algebra}

\subsection{Linear Space}
Let $\mathfrak K$ be a field.
A $\mathfrak K$-linear space is an abelian group $L$,
such that there is a map $\mathfrak K \times L \to L$
satisfies the associative law and distribution law.

\subsection{Vector}
$V = \mathfrak K^n$ is called a vector space of dimension $n$.

Each element of $V$ is called vector, for $v \in V$, $v$ can be
represented by $n$ elements in $\mathfrak K$, $v = (v^1, \dots, v^n)$.

Let $e_i = (0, \dots, 1, \dots, 0)$,
then $v = e_iv^i$

For vectors $v_1, \dots, v_n \in V$, and elements $a^1, \dots, a^n \in \mathfrak K$,
$v_1a^1 + \dots + v_na^n$ is simple denoted $v_ia^i$.

\subsection{Covector}
A covector is a linear map $f : V \to \mathfrak K$, all covectors is called the dual space of $V$, denoted by $V^\ast$.

Let $e_i^\ast \in V^\ast$, such that $e_i^\ast(e_j) = \delta_i^j$, then $\{e_i^\ast\}$ is a basis of $V^\ast$.

$(V^\ast)^\ast \cong V$.
$V$ and $V^\ast$ are $\mathfrak K$-linear spaces.
The tensors over $V$ are also $\mathfrak K$-linear spaces.


\subsection{Tensor}
Let $V_1, \dots V_k$ be $\mathfrak K$-linear spaces,
a map $f : V_1 \times \dots \times V_k \to \mathfrak K$ is
multilinear if $f(v_1a^1, \dots, v_ka^k) = f(v_1, \dots, v_k)a^1\cdots a^k$.
When $V_1 = \cdots = V_k = V$, we say $f$ is a covariant $k$-tensor over $V$;
when $V_1 = \cdots = V_k = V^\ast$, we say $f$ is a contravariant $k$-tensor over $V$;
when $V_1 \times \dots \times V_k = V^r \times (V^\ast)^s$,
we say $f$ is a tensor of type $\choose{r s}$.

All tensors of type $(r, s)$ is denoted by $\mathcal T^r_s(V)$,
especially $\mathcal T^k(V) = \mathcal T^k_0(V)$,
$\mathcal T_k(V) = \mathcal T^0_k(V)$.

% A covariant $k$-tensor is a tensor of type $(k, 0)$,
% a contravariant $k$-tensor if a tensor of type $(0, k)$.

For $S \in \mathcal T^k(V)$ and $T \in \mathcal T^l(V)$,
define $(S \otimes T) : V^{k + l} \to \mathfrak K$ by 
$(S \otimes T)(v_1, \dots, v_k, v_{k+1}, \dots, v_{k+l})
= S(v_1, \dots, v_k)T(v_{k+1}, \dots, v_{k+1})$.
This gives a map $\otimes : \mathcal T^k(V) \times \mathcal T^l(V) \to \mathcal T^{k+l}$,
$\otimes$ is called tensor product.
For $\sigma \in S_k$ and $T \in \mathcal T^k(V)$,
define $(T \circ \sigma)(v_1, \dots, v_k) = T(v_{\sigma(1)}, \dots, v_{\sigma(k)})$,
$T$ is a alternating form if $T = (-1)^\sigma T \circ \sigma$.

Define $\alt : \mathcal T^k(V) \to \mathcal T^k(V)$ by
$\alt(T) = {1 \over k!} \sum_{\sigma \in S_k} (-1)^\sigma T \circ \sigma$,
and let $\Lambda^k(V) = \alt(\mathcal T^k(V))$.

For $\sigma \in S_k$ and $T \in \mathcal T^k(V)$,
define $S \wedge T = {(k + l)! \over k! l!} \alt(S \otimes T)$.


\section{Differential Manifold}

\subsection{Manifold}
A manifold is an object which is locally isomorphic to an euclid space.

\subsection{Tangent Space}

$T_p(M) \cong \mathbb R^n$

$$T(M) = \coprod_{p \in M} T_p(M)$$


\subsection{Vector Field}
A vector Field $X$ assigns each $p \in M$
a tangent vector $X_p \in T_p(M)$.

All vector fields is denoted by $\mathfrak X(M)$,
which forms a Lie algebra.
For $X, Y \in \mathfrak X(M)$,

\subsection{Tensor Field}
$\mathcal T^r_{s \, p}(M) = \mathcal T^r_s(T_p(M))$

$$\mathcal T^r_s(M) = \coprod_{p \in M} \mathcal T^r_{s \, p}(M)$$

\subsection{Riemann Metric}
The Riemann metric $g$ assign each tangent space $T_p(M)$ a inner product $g_p$,
such that the function of $p$, $g(X, Y)(p) = g_p(X_p, Y_p)$ is differentiable.

\subsection{Connection}

The affine connection $\nabla : \mathfrak X(M) \times \mathfrak X(M) \to \mathfrak X(M)$,
$(X, Y) \to \nabla_X Y$, satisfies:

\end{document}
