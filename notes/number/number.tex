\documentclass{article}
\usepackage{amsmath}
\title{Elementary Number Theory}

\begin{document}
\maketitle
\newpage
\tableofcontents
\newpage

\section{Divisiblity}

\section{Modular Arithmetic}

\section{Arithmetic Function}
An arithmetic function is a function defined on the positive integers.

\subsection{Additive Function}
An arithmetic function $f$ is additive if $f(mn) = f(m)f(n)$ for $(m, n) = 1$.

\subsubsection*{Omega Functions}
For integer $n$, $\omega(n)$ counts the number of distinct prime factors of $n$,
$\Omega(n)$ countts the number of prime factors of $n$ with multiplicity.
$\omega(n) = \Omega(n)$ when $n$ is square free.

$\Omega(mn) = \Omega(m) + \Omega(n)$ for any positive integers $m$ and $n$.
This kind of additive function is called complete additive function.


\subsection{Multiplicative Function}
An arithmetic function $f$ is multiplicative if $f(mn) = f(m)f(n)$ for $(m, n) = 1$.
If has decomposition $n = \prod_i^k p_i^{e_i}$
$$f(n) = \prod_i^k f(p_i^{e_i})$$

\subsubsection*{Divisor Function}
The divisor function $d(n)$ counts the number of positive divisors of $n$.
$$d(n) = \prod_{p_i^{e_i} \vert\vert n} (1 + e_i)$$

\subsubsection*{Divisor Sum Function}
The divisor sum function $\sigma(n)$ is the sum of all positive divisors of $n$.
\begin{equation*}
\sigma(n) = \prod_{{p^e} \vert\vert n} (1 + p + \cdots + p^e)
= \prod_{p_i^{e_i} \vert\vert n} {p^{e + 1} - 1 \over p - 1}
\end{equation*}
In general, we can define $\sigma_k(n)$ to be the sum of $k$-th exponent of all
positive divisors of $n$ then $d(n) = \sigma_0(n)$ and $\sigma(n) = \sigma_1(n)$
become special cases.
\begin{align*}
\sigma_k(n) &= \prod_{p^e \vert\vert n} (1 + p^k + \cdots + p^{ke}) \\
\sigma_k(n) &= \prod_{p^e \vert\vert n} {p^{k(e + 1)} - 1 \over p^k - 1} \, (k \neq 0)
\end{align*}

\subsubsection*{Euler's Totient Function}
Euler's totient function $\varphi(n)$ counts the numebr of positive integers less or equal than
$n$ and relative prime to $n$.
$$\varphi(n) = n \prod_{p \vert n} (1 - {1 \over p})$$


\subsubsection*{M\"obius Function}
For power of prime number $p^e$ we define
\begin{equation*}
\mu(p^e) =
\begin{cases}
  -1 & \quad (e = 1) \\
  0  & \quad (e > 1)
\end{cases}
\end{equation*}
$\mu$ extends to a multiplicative function naturally, where $\mu(n) = 0$ if and only if $n$
is a multiple of prime square, $\mu(n)$ counts the parity of the number of distinct prime
factors of $n$ when $n$ is square free.
\begin{equation*}
\mu(n) =
\begin{cases}
  (-1)^{\omega(n)} = (-1)^{\Omega(n)} \quad & (\omega(n) = \Omega(n)) \\
  0 & (\omega(n) < \Omega(n))
\end{cases}
\end{equation*}

\subsection{Ring of Arithmetic Functions}
\subsubsection*{Dirichlet Convolution}
For two arithmetic functions $f$ and $g$, their Dirichlet convolution is another arithmetic function
define by $$h(n) = \sum_{d \vert n} f(d) g({n \over d})$$

\subsubsection*{M\"obius Transform}
For an arithmetic function $f(n)$, it's M\"obius transform is another arithmetic function
defined by $$g(n) = \sum_{d \vert n} f(d)$$
$f(n)$ can be recovered from $g(n)$ by the M\"obius inversion formula
$$f(n) = \sum_{d \vert n} \mu({n \over d}) g(d)$$

\end{document}
