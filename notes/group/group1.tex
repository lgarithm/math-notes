\section{Group and Subgroup}
Generally speaking, a group is a set with an associative binary operation on it,
which has identity and inverse property.
\begin{defi}[group]
A group $(G, \circ)$ is a set $G$ with a binary operation $\circ : G \times G \to G$ on it,
such that for any $x, y, z \in G$,
\begin{enumerate}[i).]
\item $x \circ (y \circ z) = (x \circ y) \circ z$;
\item there exists unique element $e \in G$ such that $e \circ x = x \circ e = x$,
such element is called the identity of $G$;
\item there exists unique $x^\prime$ such that $x \circ x^\prime = x^\prime \circ x = e$;
\end{enumerate}
\end{defi}
When the operation is known, $(G, \circ)$ is often denoted $G$ for short, $x \circ y$ will be simply denoted by $xy$
We use capital letters such as $G$, $H$ to denote groups, use $e$ to denote the identity of arbitrary group;
the symbol $1_G$ will be used as the identity of a specific group $G$. The inverse of $x$ is denoted by $x^{-1}$.
\begin{defi}[subgroup]
Let $G$ be a group and $H$ a subset of $G$, if $H$ forms a group under the operation of $G$, then we call $H$ a subgroup of $G$.
Denoted $H \leq G$.
\end{defi}
Obviously $\{e\}$ and $G$ are subgroups of $G$, we call them trivial subgroups, other subgroups are non-trivial subgroups.
\begin{pro}
Let $G$ be a group, $H \subseteq G$. Then $H \leq G$ if
$H$ is closed under the operation of $G$, and contains all inverses of its elements.
\end{pro}
\begin{prf}
For any $a, b, c \in H$, $a, b, c \in G$, so $(ab)c = a(bc)$;
$a^{-1} \in H$, so $1_G = a a^{-1} \in H$.
$1_G a = a 1_G$, $1_H = 1_G$.
\end{prf}
\begin{pro}
The intersection of subgroups is also a subgroup.
\end{pro}
\begin{prf}
Let $H_i$ be a family of subgroups, $a, b \in \bigcap H_i$,
then $\forall i, a, b \in H_i$; according to the above proposition $ab \in H_i$ and $a^{-1} \in H_i$,
so $ab \in \bigcap H_i$, $a^{-1} \in \bigcap H_i$, \qed.
\end{prf}
\begin{defi}
Let $S$ be a subset of a group $G$, the intersection of all subgroups which contain $S$ is called the subgroup generated by $S$.
\end{defi}
In fact, each element of the subgroup generated by $S$ can be represented as finite product of the elements of the union of $S$ and $S^{-1}$, where $S^{-1}$ is the set of all inverse of $S$.
\begin{defi}[generating set]
If the subgroup generated by $S$ is $G$, then we call $S$ a generating set of $G$.
\end{defi}
Group who has a single-element generating set is very important, such groups are called cyclic groups.
\begin{defi}[cyclic group]
The subgroup generated by one element is called a cyclic subgroup.
If a group has a cyclic subgroup which equals itself, this group is called a cyclic group.
\end{defi}
Group operations are associative but do not always communicative, so we have:
\begin{defi}[abelian group]
A group is communicative if for any $x, y \in G$, there is $xy = yx$.
\end{defi}
A communicative group is also call an Abelian group, after the great mathematical genius Abel.


\section{Coset Decomposition and Lagrange's Theorem}
Let $G$ be a group, $H$, $K$ be subsets of $G$ $g \in G$,
these notations will be commonly used:
$Hg=\{hg \mid h \in H\}$, $gH=\{gh \mid h \in H\}$,
and $HK=\{hk \mid h \in H, k \in K\}$.
\begin{pro}
\begin{enumerate}[i).]
\item $Hg=H\{g\}$;
\item $\abs{Hg}=\abs{gH}=\abs{H}$;
\item $(HK)M=H(KM)$;
\item $HH = H$.
\end{enumerate}
\end{pro}
Notice that though $\abs{Hg}=\abs{H}$,
but $\abs{HK}$ do not always equal to $\abs{H}\abs{K}$.
\begin{defi}[coset of subgroup]
Let $H \leq G, g \in G$, then $Hg$ is called a right coset of $H$
while $gH$ is called a left coset of $H$.
\end{defi}
\begin{pro}
For any two cosets of a subgroup $H$ of $G$,
they are either equal or disjoint.
\end{pro}
\begin{prf}
We only proof for right coset.
Each right coset of $H$ has the form $Hg$.
For two cosets $Hg_1$ and $Hg_2$, if $x \in Hg_1 \cap Hg_2$,
then $\exists h_1, h_2$, such that $x=h_1g_1=h_2g_2$,
so $g_1=h_1^{-1}h_2g_2$, $Hg_1=Hh_1^{-1}h_2g_2=Hg_2$.
\end{prf}
Obviously derived from the two propositions above,
we have the Lagrange's theorem for finite groups.
\begin{thm}[Lagrange]
Let $H \leq G$, then $\abs{H} \vert \abs{G}$.
\end{thm}
In fact, a group is the disjoint union of all cosets of a subgroup,
in other words, if $H \leq G$, all cosets of $H$ form a partition of $G$.
\begin{defi}[index of a subgroup]
The index of a subgroup $H$ of $G$, denoted by $[G:H]$, is defined as the number of cosets(may be infinite) of $H$ in $G$.
\end{defi}
Obviously when $G$ is finite, $[G:H] = \abs{G} / \abs{H}$.


\section{Normal Subgroup and Quotient Group}
\begin{defi}[normal subgroup]
A subgroup $H$ of $G$ is normal if $Hg = gH$ for all $g \in G$, denoted $H \lhd G$.
\end{defi}
Since there is no difference between left coset and right coset of a normal subgroup, we just call them coset, regardless of left or right.
\begin{pro}
Let $H \lhd G$, and $U$, $V$ be cosets of $H$, then
\begin{enumerate}[i).]
\item $UV$ is also a coset of $H$.
\item $HU=UH=U$;
\item there exist a coset $W$ such that $UW=WU=H$;
\end{enumerate}
\end{pro}
\begin{prf}
Let $U = Hg_1, V = Hg_2$, then $UV = Hg_1Hg_2 = HHg_1g_2$. $\qed$
\end{prf}
Derived from the propositions above, we soon realize that all cosets of $H$ forms a group, we call this group the quotient group of $G$ modulo $H$, denoted $G/H$.
Obviously $\{e\}$ and $G$ are normal subgroups of $G$. But not all groups contain non-trivial normal subgroup.
\begin{defi}[simple group]
The group which does not contain non-trivial subgroups is called a simple group.
\end{defi}
Say, the group who has exact $p$ elements is a simple group, where $p$ is a prime.
\begin{defi}
For $a, b \in G$, define the commutor of $a, b$ to be $[a,b]=aba^{-1}b^{-1}$.
\end{defi}
\begin{pro}
\begin{enumerate}[i).]
\item $[a,b]^{-1}=[b,a]$;
\item $g[a,b]g^{-1}=[gag^{-1},gbg^{-1}]$.
\end{enumerate}
\end{pro}
\begin{defi}
$G^\prime$ is the subgroup generated by $\{[a,b] \mid a, b \in G\}$.
\end{defi}
\begin{thm}
$G^\prime$ is normal subgroup of $G$.
\end{thm}
\begin{prf}
We prove this by showing $gG^\prime g^{-1} = G^\prime$ for any $g \in G$.
$\forall x \in G^\prime$, $x = [a_1, b_1][a_2, b_2] \dots [a_n, b_n]$,
\begin{align}
gxg^{-1}
&= g[a_1, b_1][a_2, b_2] \dots [a_n, b_n]g^{-1} \nonumber\\
&= g[a_1, b_1]g^{-1}g[a_2, b_2]g^{-1}g \dots g^{-1}g[a_n, b_n]g^{-1} \nonumber\\
&= [ga_1g^{-1}, gb_1g^{-1}][ga_2g^{-1}, gb_2g^{-1}] \dots [ga_ng^{-1}, gb_ng^{-1}] \in G^\prime \nonumber
\end{align}
$\qed$
\end{prf}


\section{Homomorphism and Isomorphism}
\begin{defi}[homomorphism]
Let $G$ and $H$ be groups and $f : G \to H$ is a map, if $\forall x, y \in G, f(xy)=f(x)f(y)$ , $f$ is called a homomorphism from $G$ to $H$.
\end{defi}
\begin{pro}
Let $f$ be homomorphism from $G$ to $H$, then
\begin{enumerate}[i).]
\item $f(1_G)=1_H$;
\item $f(g^{-1})=f(g)^{-1}$.
\end{enumerate}
\end{pro}
\begin{defi}[isomorphism]
If a homomorphism from $G$ to $H$ is a one-to-one map, we call it an isomorphism between $G$ and $H$.
And say $G$ is isomorphic to $H$, write $G \cong H$.
\end{defi}
Isomorphism is an equivalent relation between groups.
All homomorphisms from a group $G$ to itself is denoted by $End(G)$,
all isomorphisms from a group $G$ to itself is denoted by $Aut(G)$.
For example, for each $g$ of group $G$, the map $\sigma_g : G \to G$, $\sigma_g(x) = gx$, is an autoisomorphism of $G$.
\begin{pro}
$Aut(G)$ is a group under the operation of map connection.
\end{pro}
\begin{defi}
The inner autoisomorphism group $G$, $In(G)$, is consist of all $f_g$.
\end{defi}
\begin{pro}
$In(G)$ is a normal subgroup of $Aut(G)$.
\end{pro}
\begin{prf}
Observe the fact $\pi \circ \sigma_g \circ \pi^{-1} = \sigma_{\pi(g)}$. $\qed$
\end{prf}
\begin{defi}
The outer autoisomorphism group of $G$ is the quotient group $Aut(G)/In(G)$.
\end{defi}
There are some important theorems about homomorphism.
\subsection{Homomorphism theorems}
\begin{defi}[kernal]
The kernel of a homomorphism $f : G \to H$ is the subset of $G$ which is mapped to $1_H$ under $f$, denoted $\ker f$, $\ker f=\{g \in G \mid f(g)= 1_H\}$.
\end{defi}
\begin{pro}
A homomorphism $f$ is an injection if $\ker f = \{1_G\}$.
\end{pro}
\begin{lem}
$\ker f$ is a normal subgroup of $G$.
\end{lem}
\begin{thm}
Let $f$ be a homomorphism from $G$ to $H$, then $G / \ker f \cong \im f$.
\end{thm}
\begin{prf}
The map $p: G / \ker f \to \im f$, $p(g\ker f) = f(g)$ gives the isomorphism.
\end{prf}
This theorem is called the first theorem.


\section{Conjugate and Center}
\begin{defi}
Let $G$ be a group, two elements $a, b$ of $G$ are callecd conjugate if
exist an element $g$ of $G$ such that $a = gbg^{-1}$.
\end{defi}
Conjugate is an equivalent relation on $G$.


\section{Order}
We discuss some counting properties of finite group in this section.
\begin{defi}[order of a group]
A group is a finite group if it has finite element, and the number of elements of $G$ is called the order of $G$.
\end{defi}
\begin{defi}[order of an element]
For an element $g$ of group $G$, if there exist an positive integer $d$ such that $g^d =1_G$, then $g$ is said to have order $d$, otherwise we say $g$ has infinite order.
\end{defi}
\begin{pro}
Each element of a finite group has finite order;
the order of cyclic group generated by $g$ equals the order of $g$;
the order of each element of $G$ is a divisor of $\abs{G}$.
\end{pro}
\begin{thm}
There is an isomorphism from the cyclic group of order $n$, denoted $C_n$, to the additional group $\mathbb{Z}/n\mathbb{Z}$, and there is an isomorphism from infinite cyclic group to $\mathbb{Z}$.
\end{thm}
For cyclic groups, we have some basic results, which can be derived from number theory.
\begin{thm}
A group of order $n$ is a cyclic group if and only if there is exact one subgroup of order $d$ for each divisor $d$ of $n$.
\end{thm}


\section{Group of Permutations}
A permutation is the arrangement of some objects in a line with a specific order. To be simple, we consider when the objects are natural numbers starts from one first. Since a permutation is uniquely decided by the positions of each object in the permutation, we give the definition below:
\begin{defi}
An $n$-permutation is a one-to-one map from $\{1, 2 \dots n\}$ to $\{1, 2 \dots n\}$.
\end{defi}
For arbitrary finite set $X$, we have the definition:
\begin{defi}
A permutation over a finite set $X$ is a one-to-one map from $X$ to $X$.
\end{defi}
For any two permutations $f$ and $g$ over $X$, we define $(f \circ g)(x)=f(g(x))$ for all $x \in X$.
Obviously all permutations over $X$ form a group, we call it the symmetric group of $X$ and denote it by $S_X$.
A group of permutations, or permutation group is a subgroup of some symmetric group.
\begin{thm}[Cayley]
Every finite group is isomorphic to some permutation group.
\end{thm}
\begin{prf}
Let $G$ be a finite group, consider $\phi : G \to S_G$, $\phi(g) = \sigma_g$, where $\sigma_g : G \to G$, $\sigma_g(x) = gx$ is a permutation on $G$.
Observe $\ker \phi = \{1_G\}$ so $\phi$ is a single homomorphism, therefore $G \cong \im \phi$.
\end{prf}
Notice that each permutation is a function.
Now we develop some properties of permutations.

\section{Group Action}
\begin{defi}
Let $X$ be a set and $G$ is a group, the action of $G$ on $X$ is a homomorphism from $G$ to $S_X$.
\end{defi}
If group $G$ acts on $X$, then we can use the elements of $g$ as permutation on $X$. Sometimes we simply write $gx$ instead of $g(x)$.
\begin{defi}[orbit]
Suppose group $G$ acts on $X$, and $x \in X$, the orbit of $x$, denoted by $O(x)$, is defined as $\{y \mid y=gx, g \in G\}$.
\end{defi}
\begin{pro}
Each pair of oribits is either equal or disjoint.
\end{pro}
\begin{defi}[stabliser]
Suppose group $G$ acts on $X$, and $x \in X$, the stabliser of $x$ is the subset of $G$ which fix $x$, $G_x =\{g \in G \mid gx=x\}$.
\end{defi}
\begin{pro}
The stabliser of $x$ is a subgroup of $G$, but not always a normal subgroup.
\end{pro}
\begin{pro}
If $y \in O(x)$, then $\abs{G_x}=\abs{G_y}$.
\end{pro}
\begin{thm}
$\abs{O(x)}=[G:G_x]$
\end{thm}
\begin{prf}
We prove by construct a one-to-one map from $O(x)$ to all cosets of $G_x$. Let $f(gx)=gG_x$, it’s not hard to verify f is a one-to-one map.
\end{prf}
\begin{defi}[fixed point]
Suppose group $G$ acts on $X$, the fixed points of $g$ is the set of all elements of $X$ remain unchanged under transform $g$,
denoted $\Fix(g)=\{x \in X \mid gx=x\}$, and we use $\fix(g)$ to denote the number of fixed points of $g$, i.e. $\fix(g)=\abs{\Fix(g)}$.
\end{defi}
\begin{thm}[non-burnside]
Suppose group $G$ acts on $X$, and $N$ is the number of orbits, then $$\abs{G}N = \sum_{g \in G} \fix(g)$$
\end{thm}\section{Free Group}
\begin{defi}[word]
Let $X$ be a set, a word over $X$ is a sequence of zero or finite length and all items in the sequence are elements of $X$, all words over $X$ is denoted by $X^\ast$, a word of zero length is called an empty word.
\end{defi}
\begin{pro}
For a non-empty set $X$, $X^\ast$ is an infinite set.
\end{pro}
When we talk words, the set $X$ is always called an alphabet and the element of $X$ is called a letter. For example, each simple English word is a word of the Latin alphabet.
\begin{defi}[concatenation of words]
If $s=s_1s_2 \dots s_m$ , $t=t_1t_2 \dots t_n$ are two words over $X$, obviously $s_1s_2 \dots s_mt_1t_2 \dots t_n$ is also a word over $X$, we call it the concatenation of $s$ and $t$, and denote it by $st$.
\end{defi}
\begin{pro}
If $s$ and $t$ are two words, the length of $st$ equals the sum of length of $s$ and $t$.
\end{pro}
In order to define free group, we must introduce the concept of inverse letter and inverse alphabet.
\begin{defi}
If $x$ is a letter of $X$, we define a special letter $x^\prime$, called the inverse letter of $x$, such that the concatenation of $x$ and $x^\prime$ is the empty word, and the inverse letters of each letter are different.
\end{defi}
\begin{defi}
Let $X$ be an alphabet, the inverse alphabet of $X$ is a set $X^\prime$, consist of all inverse letters of elements of $X$.
\end{defi}
\begin{defi}[free group]
Let $X$ be an alphabet, and $X^\prime$ be its inverse alphabet, we denote the word over the union $X$ and $X^\prime$, $F(X)$, we can see that $F(X)$ is a group under the operation of word concatenation.
\end{defi}
