\section{Basic Notations of Set}
Let $X$ be a set, then $2^X, \ps{X}$ will denote the power set of $X$;
$A \subseteq X$ if and only if $A \in \ps{X}$, such $A$ is called a subset of $X$.
We call a set contains only one element a singleton set,
and a set contains no elements an empty set, denoted by $\emptyset$.

Union $\cup$ and intersection $\cap$ are maps $\ps{X}^2 \to \ps{X}$ defined by:
$$A \cup B = \{x \mid x \in A\mbox{ or }x \in B\}$$
$$A \cap B = \{x \mid x \in A\mbox{ and }x \in B\}.$$
When $X$ is well known, for a subset $A$ of $X$, define the complement of $A$ by:
$$A^c = \{x \mid x \not\in A\}$$

\section{Algebra of Subsets}
\begin{defi}
Let $\mathcal{A} = \{A_i\}_{i \in I}$ be a family of subsets of $X$,
define
$$\bigcup \mathcal{A} = \bigcup_{i \in I} A_i = \{x \mid \exists i \in I, x \in A_i\}$$
and
$$\bigcap \mathcal{A} = \bigcap_{i \in I} A_i = \{x \mid \forall i \in I, x \in A_i\}.$$
\end{defi}
If $\mathcal{A} = \emptyset$, we make a convention that $\bigcup \mathcal{A} = \emptyset$
and $\bigcap \mathcal{A} = X$.

\begin{defi}
Let $\mathcal{C} \subseteq \ps{X}$, if $\bigcup \mathcal{C} = X$, 
then call $\mathcal{C}$ a cover of $X$;
if the intersection of any two distinct elements of a cover is the empty set,
then call it a partition.
\end{defi}

Let $f : X \to Y$ be a function, for $U \subseteq X$ and $V \subseteq Y$,
define the image of $U$ and pre-image of $V$ by
$$f(U) = \{f(x) \mid x \in X\}$$
$$f^{-1}(V) = \{x \in X \mid f(x) \in V\}.$$

\begin{pro}
\begin{enumerate}[i).]
\item $f(f^{-1}(V)) = V$;
\item $f^{-1}(f(U)) \supseteq U$;
\item $f^{-1}(V_1 \cup V_2) = f^{-1}(V_1) \cup f^{-1}(V_2)$.
\end{enumerate}
\end{pro}

\section{Topology Space}
\begin{defi}
A topology $\mathcal{T}$ on $X$ is a subset of $\mathcal{P}(X)$, such that
\begin{enumerate}[i).]
\item $\emptyset, X \in \mathcal{T}$;
\item if $\{U_i\}_{i=1}^n \subseteq \mathcal{T}$, then $U_1 \cap U_2 \cap \dots \cap U_n \in \mathcal{T}$;
\item if $\{U_i\}_{i \in I} \subseteq \mathcal{T}$, $\bigcup_{i \in I} U_i \in \mathcal{T}$.
\end{enumerate}
\end{defi}
If $\mathcal{T}$ is a topology on $X$, the pair $(X, \mathcal{T})$ is called a topology space.

\begin{pro}
If $\mathcal{S}, \mathcal{T}$ are topologies on $X$, then $S \cap T$ is also a topology on $X$;
if $\mathcal{T}_i \, (i \in I)$ are topologies on $X$, then $\bigcap_{i \in I} \mathcal{T}_i$ is also a topology on $X$.
\end{pro}

If $\mathcal{S}, \mathcal{T}$ are topologies on $X$, and $\mathcal{S} \subseteq \mathcal{T}$,
% then $\mathcal{S}$ is called a subtopology of $\mathcal{T}$.
we say $\mathcal{T}$ is finer than $\mathcal{S}$,
and $\mathcal{S}$ is coarser than $\mathcal{T}$.

\begin{defi}
Let $\mathcal{B} \subseteq \ps{X}$, the topology generated by $\mathcal{B}$
is the intersection of all topologies on $X$ that contains $\mathcal{B}$.
\end{defi}

\begin{defi}
Let $\mathcal{T}$ be a topology on $X$,
an element of $\mathcal{T}$ is called an open set;
The complement of an open set if called a closed set.
\end{defi}

\begin{exer}
Prove that the finite union and arbitrary intersection of closed sets are closed.
\end{exer}

\section{Basis}
\begin{defi}
Let $\mathcal{B} \subseteq \ps{X}$, if $\{\bigcup \mathcal{A} \mid \mathcal{A} \subseteq \mathcal{B}\}$
is a topology on $X$, then call $\mathcal{B}$ a basis on $X$.
\end{defi}
A basis is sufficiently a cover, for there must be $\bigcup \mathcal{B} = X$.
But the reverse is not true.

\begin{pro}
A cover $\mathcal{B}$ is a basis if and only if
for any $B_1, B_2 \in \mathcal{B}$ and $x \in B_1 \cap B_2$,
exist $B_3 \in \mathcal{B}$ such that $x \in B_3 \subseteq B_1 \cap B_2$.
\end{pro}
If $\mathcal{B}$ is a basis, the topology generated by $\mathcal{B}$
equals $\{\bigcup \mathcal{A} \mid \mathcal{A} \subseteq \mathcal{B}\}$.
If $\mathcal{B}$ generates $\mathcal{T}$, we call $\mathcal{B}$ a basis of $\mathcal{T}$.

\section{Subbasis}
\begin{defi}
Let $\mathcal{C} \subseteq \ps{X}$, $\mathcal{C}$ is a subbasic on $X$
if $\{\bigcap \mathcal{S} \mid \mathcal{S} \subseteq \mathcal{C}\mbox{ is a finite subset}\}$
is a basis on $X$.
\end{defi}
When $\mathcal{S} = \emptyset$, $\bigcap \mathcal{S} = X$,
therefore subbasis $\mathcal{C}$ need not to be a cover.
A partition is a subbasis.

\begin{pro}
A cover is a subbasis.
\end{pro}
A topology is a basis; a basis is a cover; a cover is a subbasis.
If $\mathcal{C}$ is a subbasis, the topology generated by $\mathcal{C}$
equals the topology generated by the basis
$\{\bigcap \mathcal{S} \mid \mathcal{S} \subseteq \mathcal{C}\mbox{ is a finite subset}\}$.
If $\mathcal{C}$ generates $\mathcal{T}$, we call $\mathcal{C}$ a subbasis of $\mathcal{T}$.

\section{Subspace}
\begin{pro}
Let $(X, \mathcal{T})$ be a topology space, $Y \subseteq X$ a subset of $X$,
then $\mathcal{S} = \{U \cap Y \mid U \in \mathcal{T}\}$ is a topology on $Y$.
\end{pro}
$(Y, \mathcal{S})$ is called a subspace of $(X, \mathcal{T})$.

\begin{pro}
Let $\mathcal{B}$ be a basis of $\mathcal{T}$,
then $\{B \cap Y \mid B \in \mathcal{B}\}$ is a basis of $\mathcal{S}$.
\end{pro}
\begin{pro}
Let $\mathcal{C}$ be a subbasis of $\mathcal{T}$,
then $\{C \cap Y \mid C \in \mathcal{C}\}$ is a subbasis of $\mathcal{S}$.
\end{pro}

\section{Neibourhood}
\begin{defi}
Let $(X, \mathcal{T})$ be a topology space,
$N \subseteq X, x \in X$, $N$ is a neibourhood of $x$ if
exist open set $U$, $x \in U \subseteq N$.
\end{defi}
If $N$ is a neibourhood of $x$ and $N$ is open, we call $N$ an open neibourhood of $x$.

\begin{pro}
Let $(X, \mathcal{T})$ be a topology space, then
\begin{enumerate}[i).]
\item $\{N \mid N\mbox{ is a neibourhood of }x\} \cup \{\emptyset\}$ is a topology on $X$;
\item $\{N \mid N\mbox{ is an open neibourhood of }x\} \cup \{\emptyset\}$ is a topology on $X$,
% which is a subtopology of $\mathcal{T}$.
which is coarser than $\mathcal{T}$.
\end{enumerate}
\end{pro}


\section{Interior}
\begin{defi}
Let $(X, \mathcal{T})$ be a topology space, $A \subseteq X$,
the interior of $A$ is the union of all open sets that contained in $A$,
denoted by $A^\circ$ or $\mathop{\mathrm{Int}} A$.
\end{defi}
From definition we know that $A^\circ$ must be open,
$A^\circ \subseteq A$, therefore $A$ is open if and only if $A = A^\circ$.
\begin{pro}
For any $x \in A^\circ$, exist open set $U$,
such that $x \in U \subseteq A$.
\end{pro}
\begin{defi}
Let $x \in A \subset X$, if there is an open set $U$
such that $x \in U \subseteq A$, then call $x$ an interior point of $A$.
\end{defi}
From definition we know that $x$ is an interior point of $A$ if and only if $A$ is a neibourhood of $x$.
\begin{pro}
$A^\circ = \{x \mid x\mbox{ is an interior point of }A\}$.
\end{pro}


\section{Closure}
\begin{defi}
Let $(X, \mathcal{T})$ be a topology space, $A \subseteq X$,
the closure of $A$ is the intersection of all closed sets that contains $A$,
denoted by $\bar A$, or $\mathop{\mathrm{Cl}} A$.
\end{defi}
$\bar A$ is closed, $A \subseteq \bar A$, $A$ is closed if and only if $A = \bar A$.
\begin{pro}
\begin{enumerate}[i).]
\item $\bar A = ((A^c)^\circ)^c$;
\item $A^\circ = (\overline{A^c})^c$.
\end{enumerate}
\end{pro}
\begin{pro}
$x \in \bar A$ if and only if for any neibourhood $N$ of $x$, $N \cap A \neq \emptyset$.
\end{pro}


\section{Limit Point}
\begin{defi}
Let $(X, \mathcal{T})$ be a topology space, $A \subseteq X$,
$x \in X$ is a limit point of $A$ if for any neibourhood $N$ of $x$,
$(N \backslash \{x\}) \cap A \neq \emptyset$.
\end{defi}
The set of all limit points of $A$ is denoted by $A^\prime$,
call the derivative set of $A$.
\begin{pro}
$A \cup A^\prime = \bar A$.
\end{pro}


\section{Product Space}
\begin{defi}
Let $\{X_i\}_{i \in I}$ be a family of topology spaces,
$\prod_{i \in I} X_i$ is the space generated by subbasis
$\mathcal{S} = \bigcup_{i \in I} \{\pi^{-1}(U) \mid U \in \mathcal{T}_{X_i}\}$
\end{defi}


\section{Sequence}
\begin{defi}
Let $(X, \mathcal{T})$ be a topology space, $x \in X$,
$\{x_n\}$ a sequence of $X$, $\{x_n\}$ converges to $x$
if for any neibourhood $N$ of $x$, exist $n_0 \in \mathbb{N}$,
such that $x_n \in N$ for all $n \geq n_0$.
If $\{x_n\}$ converges to $x$, we call $x$ a limit point of $\{x_n\}$.
\end{defi}
\begin{defi}
A sequence $\{x_n\}$ converges if it has at least one limit point.
\end{defi}
\begin{pro}
\end{pro}


\section{Quotient Space}
\begin{defi}
Let $(\mathcal{T}, X)$ be a topology space, $Y = \mathcal{S}$ a partition of $X$,
then we defined a topology on $Y$,
for which $\mathcal{O} \subseteq \mathcal{S}$ is open if and only if $\bigcup \mathcal{O} \in \mathcal{T}$.
\end{defi}
