
\section{Linear Groups}
Let $R$ be a ring, the matrices over $R$ of order $n$, $M_n(R)$ forms a ring.

\subsection{General Linear Group}
Let $K$ be a field, the general linear group $GL_n(K)$ is the set of all
matries over $K$ of order $n$ with non-zero determinant, it forms a group under matrix multiplication.

If $V$ is a $n$-dimensional $K$-linear space, we denote $\GL(V)$ as the set of all
invertable linar transformations $V \to V$, it forms a group under composition of 
linear transformations.

Since we have $V \cong K^n$, therefore $\GL_n(K) \cong \GL(V)$.

When $R$ is a ring, the notation $\GL_n(R)$ also make sense, but to make it a group, 
we should change the condition non-zero determinant to invertable.

\subsection{Special Linear Group}
The special linear group $\SL_n(K)$ is a subgroup of $\GL_n(K)$, forms by the matrices
with determinant $1$.

When $R$ is a ring, $\SL_n(R)$ still forms a group.

\subsection{Orthogonal Group}
A matrix $M$ is orthogonal if $M^T M = M M^T = I$, where $M^T$ is the transpose of $M$.

The orthogonal group $O_n(K)$ is the set of all orthogonal matrices, it is a subgroup of $GL_n(K)$.

The special orthogonal group $SO_n(K)$ is a subgroup of $O_n(K)$.

\subsection{Unitary Group}
A complex matrix $M$ is unitary if $M^\ast M = M M^\ast$, where $m^\ast$ is the conjugate transpose of $M$.

The unitary group $U_n$ is the set of all unitary matrices, it is a subgroup of $\GL_n(\mathbb C)$.

The special unitary group $SU_n$ is a subgroup of $U_n$.

\subsection{Symplectic group}
The symplectic group, $\Sp$
