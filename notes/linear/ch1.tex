\section{Elementary Theory}
Before enter the abstract concepts of linear algebra,
we first give some concrete examples on the objects of linear algebra,
the vectors and the matrices.

The definitions given in this section are totally constructive.

\subsection{Vector}
A vector of length $n$ is an array of $n$ numbers $v = (a_1, \dots, a_n)$,
where $a_i$ is called the $i$-th component of $v$.
Two vectors of the same length can be added, component by component:
$$(a_1, \dots, a_n) + (b_1, \dots, b_n) = (a_1 + b_1, \dots, a_n + b_n)$$
A vector can be multipled by a number,
$$\lambda(a_1, \dots, a_n) = (\lambda a_1, \dots, \lambda a_n)$$
result in a new vector with each component multiplied by that number.

For two vectors $u = (a_1,\dots, a_n)$ and $v = (b_1,\dots, b_n)$ of the same length,
we define the inner product of them as a single number $\sum_{i=1}^n a_ib_i$.

\subsection{Matrix}
An $m$-by-$n$ matrix is $mn$ numbers arranged as $m$ rows and $n$ colomns:
$$\left(\begin{matrix}
a_{11} & a_{12} & \cdots & a_{1n} \cr
a_{21} & a_{22} & \cdots & a_{2n} \cr
\vdots & \vdots & \ddots & \vdots \cr
a_{m1} & a_{m2} & \cdots & a_{mn}
\end{matrix}\right)$$
For two matrix of the same scale, we can define their sum as termwise sum
$$\left(\begin{matrix}
a_{11} & a_{12} & \cdots & a_{1n} \cr
a_{21} & a_{22} & \cdots & a_{2n} \cr
\vdots & \vdots & \ddots & \vdots \cr
a_{m1} & a_{m2} & \cdots & a_{mn}
\end{matrix}\right) +
\left(\begin{matrix}
b_{11} & b_{12} & \cdots & b_{1n} \cr
b_{21} & b_{22} & \cdots & b_{2n} \cr
\vdots & \vdots & \ddots & \vdots \cr
b_{m1} & b_{m2} & \cdots & b_{mn}
\end{matrix}\right)$$
$$=\left(\begin{matrix}
a_{11} + b_{11} & a_{12} + b_{12} & \cdots & a_{1n} + b_{1n} \cr
a_{21} + b_{21} & a_{22} + b_{22} & \cdots & a_{2n} + b_{2n} \cr
\vdots & \vdots & \ddots & \vdots \cr
a_{m1} + b_{m1} & a_{m2} + b_{m2} & \cdots & a_{mn} + b_{mn}
\end{matrix}\right)
$$
For a $k$-by-$m$ matrix $A = (a_{ij})$ and an $m$-by-$n$ matrix $B = (b_{ij})$,
we can define their product $AB$ as a $k$-by-$n$ matrix $C = (c_{ij})$,
$$c_{ij} = \sum_{k=1}^m a_{ik}b_{kj}$$
One can check by definition $$A(BC) = (AB)C$$
and $$A(B + C) = AB + AC \quad (A + B)C = AB + BC$$
if their scale are proper to be added and multiplied.

If $A = (a_{ij})$ is an $m$-by-$n$ matrix,
we define the transpose of $A$ as a $n$-b-$m$ matrix,
$A^\prime = (a^\prime_{ij})$, where $a^\prime_{ij} = a_{ji}$.

\subsection{Square Matrix}
Now we consider $n$-by-$n$ matrix. If the terms of the matrix are taken from $K$,
we will denoted the set of all such matrix by $M_n(K)$. Usually $K$ will be a field,
or a ring. Most frequently, $\mathbb{R}$ or $\mathbb{C}$.
From early observations we notice that $M_n(K)$ form a non-commutative ring,
the $0$ and $1$ in this ring are the zero matrix
$$O = \left(\begin{matrix}
0 & 0 & \cdots & 0 \cr
0 & 0 & \cdots & 0 \cr
\vdots & \vdots & \ddots & \vdots \cr
0 & 0 & \cdots & 0 \cr
\end{matrix}\right)$$
and the identity matrix
$$I = \left(\begin{matrix}
1 & 0 & \cdots & 0 \cr
0 & 1 & \cdots & 0 \cr
\vdots & \vdots & \ddots & \vdots \cr
0 & 0 & \cdots & 1
\end{matrix}\right)$$

A square matrix $U$ is invertible, if exists $V$ such that $UV = VU = I$.
Such $V$ is called the inverse of $U$, later we will see it's unique if exists.

\subsection{Determinant}
For a $n$-by-$n$ square matrix $A = (a_{ij})$, we define its determinant as a number
$$\det A = \sum_{\pi \in S_n} a_{1\pi(1)} a_{2\pi(2)} \cdots a_{n\pi(n)}$$
Obviously observed from definition $\det A = \det A^\prime$.

The $(i, j)$ minor of $A$, denoted by $M_{ij}$ is the determinant of the square matrix
obtained by deleting $i$-th row and $j$-th column of $M$.
The $(i, j)$ algebraic cofactor of $A$ is $A_{ij} = (-1)^{i + j} M_{ij}$.
The adjoint matrix of $A$, denoted by $A^\ast$ is a matrix whose $j$-th element on $i$-th
row, $a^\ast_{ij}$ is $A_{ji}$.

\begin{pro}
$$\sum_{j=1}^n a_{ij} A_{ij} = \sum_{i=1}^n a_{ij} A_{ij} = \det A$$
$$\sum_{i=1}^n a_{ij} A_{ik} = \sum_{i=1}^n a_{ji} A_{ki} = 0 \, (j \neq k)$$
\end{pro}

\begin{cly}
$$A A^\ast = A^\ast A = (\det A) I$$
\end{cly}

\begin{pro}
$\det (AB) = (\det A)(\det B).$
\end{pro}

\begin{pro}
$M$ is invertible if and only if $\det M \neq 0$,
the inverse is ${1 \over \det M} M^\ast$.
\end{pro}

\subsection{Trace}
For $A = (a_{ij})\in M_n(K)$, we define its trace by $\tr A = \sum_{i=1}^n a_{ii}$.

\begin{pro}
If $\det U \neq 0$, then $\tr A = \tr (UAU^{-1})$.
\end{pro}

\begin{prf}
Let $A = (a_{ij}), U = (u_{ij}), U^{-1} = (v_{ij})$, then
\begin{align*}
\tr (UAU^{-1}) & = \sum_{i=1}^n \sum_{j=1}^n \sum_{k=1}^n u_{ij} a_{jk} v_{ki} \\
& = \sum_{j=1}^n \sum_{k=1}^n a_{jk} \sum_{i=1}^n v_{ki} u_{ij} \\
& = \sum_{j=1}^n \sum_{k=1}^n a_{jk} \delta_{kj} \\
& = \tr A
\end{align*}
\end{prf}
