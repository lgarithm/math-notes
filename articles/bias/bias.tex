\documentclass{article}
\usepackage{amsmath}
\title{Why We Should Divide by $n - 1$\\When Estimate the Variance}
\author{Lgarithm\\ Department of Computer Science, SCU \\\mbox{lgarithm@gmail.com}}
\date{December 23, 2011}

\begin{document}
\maketitle

Let $X = \{X_i \mid i = 1, \dots, N \}$ be the population,
then their average and variance are given by

\begin{align}
\bar X = {1 \over N}\sum_{i=1}^N X_i
\end{align}

\begin{align}
\sigma^2
 &= {1 \over N} \sum_{i=1}^N (X_i - \bar X)^2	\nonumber\\
 &= {1 \over N} (\sum_{i=1}^N X_i - 2\bar X\sum_{i=1}^N X_i + N \bar X)	\nonumber\\
 &= {1 \over N} \sum_{i=1}^N X_i - {1 \over N^2} \sum_{i=1}^N X_i^2	
\end{align}

Let $\{x_i \in X \mid i = 1, \dots, n \}$ be the samples,
then the sample average and sample variance are give by

\begin{align}
\bar x = \sum_{i=1}^n x_i
\end{align}

\begin{align}
S^2(x_1, \dots, x_n)
 &= {1 \over n - 1} \sum_{i=1}^n (x_i - \bar x)^2	\nonumber\\
 &= {n \over n - 1} {1 \over n} \sum_{i=1}^n (x_i - \bar x)^2	\nonumber\\
 &= {1 \over n - 1} (\sum_{i=1}^n x_i^2 - {1 \over n}(\sum_{i=1}^n x_i)^2)
\end{align}

We shall verify that when we pick each sample indenpendently,
the expectation of $S^2$ right equals the population variance $\sigma^2$.

We write $$\sum_{x_i \in X}$$ for
$$\sum_{x_1, \dots, x_n \in \{X_1, \dots, X_N\}}$$ from now on,
there are $N^n$ terms in the summation.

When each sample $x_i$ run through the population independently, 
there are $N^n$ possibilities in total,
therefore the expectation of $S^2$ is
\begin{align}
\label{eq:result}
E(S^2)
 &= {1 \over N^n} \sum_{x_i \in X} S^2(x_1, \dots, x_n) \nonumber\\
 &= {1 \over N^n} \sum_{x_i \in X}
		{1 \over n - 1} (\sum_{i=1}^n x_i^2 - {1 \over n}(\sum_{i=1}^n x_i)^2) \nonumber\\
 &= {1 \over N^n} {1 \over n - 1}
		\sum_{x_i \in X}
			(\sum_{i=1}^n x_i^2 - {1 \over n}(\sum_{i=1}^n x_i)^2) \nonumber\\
 &= {1 \over N^n} {1 \over n - 1}
	(\sum_{x_i \in X}
		\sum_{i=1}^n x_i^2
	 -
	 {1 \over n}
	 \sum_{x_i \in X}
		(\sum_{i=1}^n x_i)^2
	)
\end{align}

We can calculate
\begin{align}
\label{eq:first}
\sum_{x_i \in X} \sum_{i=1}^n x_i^2
\end{align}
and
\begin{align}
\label{eq:second}
\sum_{x_i \in X} (\sum_{i=1}^n x_i)^2
\end{align}
separately.


For~\eqref{eq:first}, each $x_i^2$ is just $X_j^2$ for some $j \in \{1, \dots, N\}$,
there are $N^n n$ terms in the summation,
and each $X_j^2$ occurs equal times, therefore
\begin{align}
\sum_{x_i \in X} \sum_{i=1}^n x_i^2
 &= {N^n n \over N} \sum_{i=1}^N X_i^2	\nonumber\\
 &= N^{n-1} n \sum_{i=1}^N X_i^2
\end{align}


For~\eqref{eq:second}, we first look at the inner term
\begin{align}
(\sum_{i=1}^n x_i)^2 = \sum_{i=1}^n x_i^2
	+ 2 \sum_{1\leq i < j\leq n} x_i x_j
\end{align}

Therefore~\eqref{eq:second} breaks into two parts,
\begin{align}
\sum_{x_i \in X} (\sum_{i=1}^n x_i)^2
 &= \sum_{x_i \in X}
		\sum_{i=1}^n x_i^2
	+
	\sum_{x_i \in X}
		2\sum_{1\leq i < j\leq n} x_i x_j
\end{align}

We can calculate the first part as~\eqref{eq:first},
for the second part, we should notice that $x_i$ and $x_j$
run through $X = \{X_1, \dots, X_N\}$ indenpendently, so we obtain
\begin{align}
\sum_{x_i \in X} (\sum_{i=1}^n x_i)^2
 &= N^{n-1} n \sum_{i=1}^N X_i^2
	+
	2 N^{n-2} {n \choose 2} (\sum_{i=1}^N X_i)^2
\end{align}

Finally take ther result of ~\eqref{eq:first} and~\eqref{eq:second} back into~\eqref{eq:result}
\begin{align}
E(S^2)
 &= {1 \over N^n} {1 \over n - 1}
	(
		N^{n-1} n \sum_{i=1}^N X_i^2
		-
		N^{n-1} \sum_{i=1}^N X_i^2
		-
		N^{n-2} (n - 1) (\sum_{i=1}^N X_i)^2
	) \\
 &= {1 \over N^n} {1 \over n - 1}
	(
		N^{n-1} (n - 1) \sum_{i=1}^N X_i^2
		-
		N^{n-2} (n - 1) (\sum_{i=1}^N X_i)^2
	) \\
 &= {1 \over N^n}
	(
		N^{n-1} \sum_{i=1}^N X_i^2
		-
		N^{n-2} (\sum_{i=1}^N X_i)^2
	) \\
 &= {1 \over N} \sum_{i=1}^N X_i^2
	-
	{1 \over N^2} (\sum_{i=1}^N X_i)^2	\\
 &= \sigma^2
\end{align}

That's it.
\end{document}
