
\documentclass{article}
\usepackage{amsmath}
\usepackage{amssymb}

\def\ord{\mathop{\mathrm ord}}
\def\deg{\mathop{\mathrm deg}}

\begin{document}

\section{Intro}
\subsection{Sequence}
A sequence is a simply a list of elements in given order,
$$a_0, a_1, \cdots, a_n, \cdots$$
A sequence can be either finite or infinite,
in any case, we can use natural numbers to label a sequence,
and we always start the index from $0$.
Therefore we can define an infinite sequence of elements of set $X$ formally as a
function $$a : \mathbb N \to X$$
and a sequence of length $n$ of elements of set $X$ as a function
$$a : \{1, 2, \cdots, n\} \to X$$
If $a$ is a sequence, we use $a_n$ to indicate the $n$-th element.

\subsection{Operators}
For any sequence, if we pick up part of elements in it and remain the order unchanged,
we got a subsequence.
There is one special case which we pick up all elements except the first one,
we call the new sequence the shift of the original sequence.
If $b = Ea$, we have $b_n = a_{n+1}$.

For a sequence of numbers, we can define its partial sum $s = \Sigma a$,
$s_n = a_0 + \cdots + a_n$, and difference $d = \Delta a$, $d_n = a_{n+1} - a_n$.
From simple observation $\Delta \Sigma = E$

\subsection{Generating Function}
For a sequence $a$, we can associate it with a formal sum
$$a(x) = \sum_{n=0}^{\infty} a_n x^n$$

Let $s = \Sigma a$, $b = E a$, $d = \Delta a$
$$s(x) = {1 \over 1 - x} a(x)$$
$$b(x) = {a(x) - a_0 \over x}$$
$$d(x) = b(x) - a(x) = {a(x) - a_0 \over x} - a(x)$$


\section{Formal Series}
\subsection{Definition}
Let $K$ be field, or more general a ring, a formal series with coefficients in $K$ is a formal sum
$$f(x) = \sum_{n=0}^\infty f_n x^n$$
If $f(x)$ only has finite terms, which means,
the coefficient of $x^n$ is always zero when $n > N$
for some constannt $N$, we call $f$ a polynomial.
The ring of all formals series overs $K$ is denoted by $K[[x]]$,
and the set of all polynomials $K[x]$ is a subset of $K[[x]]$.

To be more rigious, we can define a formal series $f$ over $K$ as a function $f : \mathbb N \to K$,
and write the value of $f$ at $n$ as $f_n$.
Then a polynomial is just a function $f : \mathbb N \to K$ with finite support.

For $a = a_0 + a_1 x + \cdots a_nx^n + \cdots$ and $b = b_0 + b_1 x + \cdots b_n x^n + \cdots$
we can define their sum
$$a + b = (a_0 + b_0) + (a_1 + b_1) x + \cdots + (a_n + b_n) x^n + \cdots$$
and product
$$ab = a_0 b_0 + (a_0 b_1 + a_1 b_0) x + \cdots + (a_0b_n + \cdots + a_nb_0) x^n + \cdots$$
then $K[[x]]$ forms a ring and $K[x]$ becomes a subring of $K[[x]]$.

For $f \in K[[x]]^\ast$, there is a smallest index $n \in \mathbb N$ suth the coefficient 
$f_n$ is non-zero, we define this number as the order of $f$, and denote it by $\ord f$. 
$$\ord(fg) = \ord f \ord g$$

For $f \in K[x]^\ast$, there is also a largest index $n \in \mathbb N$ such that there coefficients
$f_k = 0$ when $k > n$. We define this number as the degree of $f$, and denoted it by $\deg f$.
$$\deg(fg) = \deg f \deg g$$

From the Gauss theorem we know that $R[x]$ is a domain if $R$ is a domain.
In fact we can also prove that $R[[x]]$ is a domain if $R$ is a domain.

\subsection{Inverse}
Now we consider the invertable elements in $K[[x]]$.
Assume that $ab = 1$ and $a_0 = 1$
we have
\begin{align*}
a_0 b_0 &= 1 \\
a_1 b_0 + b_1 &= 0 \\
a_2 b_0 + a_1b_1 + b_2 &= 0 \\
a_3 b_0 + a_2b_1 + a_1b_2 + b_3 &= 0 \\
& \cdots
\end{align*}
From these equations, we can have
\begin{align*}
b_0 &= 1 \\
b_1 &= -a_1 \\
b_2 &= -a_2 - a_1b_1 &= -a_2 + a_1^2 \\
b_3 &= -a_3 - a_2b_1 - a_1b_2 \\
& \cdots
\end{align*}
Therefore we can solve for $b_i$ for all $i = 0, 1, \dots$ in terms of $a_i$.

In general,
$$b_i = \sum_{j_1 + \cdots j_k = i,\\ j_r > 0} (-1)^k a_{j_1} \cdots a_{j_k}$$

Even when $K$ is a ring, we can always find an inverse for a
when $a_0 = 1$.
When $K$ is a field, the only requirement is $a_0 \neq 0$.
Obviously this condition is sufficient and necessary,
hence when $a_0 = 0$, there is no $b_0 \in K$ such that $a_0 b_0 = 1$.
That is to say, an element of $K[[x]]$ is not invertable if and only if $a_0 = 0$,
then it must be a multiple of $x$.
Therefore, all non-unit of $K[[x]]$ forms the principal ideal generated by $x$,
that is to say, $K[[x]]$ is a local ring with maximal ideal $(x)$.
And we have the natural isomorphism $K[[x]]/(x) \cong K$, where $\sum a_n x^n \to a_0$.
Notice that we don't have $K[[x]]/(x - p)$ isomorphic to $K$ when $p \neq 0$ as in $K[x]$,
for $\sum a_n p^n$ is not always well defined when there are infinite manny non-zero coefficients.


\subsection{Derivative}
For a formal series, we can define its formal derivative, without the concept of limit.
$$D(f) = \sum_{n=0} (n + 1) f_{n + 1} x^n$$
Therefore $D$ is a linear operator $K[[x]] \to K[[x]]$, and satisfies the Libniz's law
$$D(fg) = D(f) g + f D(g)$$

\section{Applications}
\subsection{Some Useful Identities}
$${1 \over 1 - x} = 1 + x + \cdots + x^n + \cdots$$
take the $k$-the derivative and then divide by $k!$ on both sides,
we got
$${1 \over (1 - x)^{1 + k}} = \sum_{n=0}^\infty {n + k\choose k} x^n$$

\subsection{Power Sum}
Let $S_k(n) = 1^k + \cdots + n^k$, the sum of $k$-th powers of the first $n$
positive integers. For $k = 1, 2, 3$, we have well known formulas:
\begin{align}
S_1(n) &= 1 + \cdots + n &= {n(n+1) \over 2} \\
S_2(n) &= 1^2 + \cdots + n^2 &= {n(n + 1)(2n + 1) \over 6} \\
S_3(n) &= 1^3 + \cdots + n^3 &= {n^2(n + 1)^2 \over 4}
\end{align}
However, when $k$ gets larger, the formula gets more complex.
But we know that the formula for $S_k$ is a $(k + 1)$-th polynomial of $n$.

In terms of generating functions, we can rewrite for formulas for them as
\begin{align*}
{1 \over 1 - x} \sum_{n=0}^\infty n x^n 
	&= \sum_{n=0}^\infty {n(n + 1) \over 2} x^n 
	&= \sum_{n=0}^\infty {n + 1 \choose 2} x^n \\
{1 \over 1 - x} \sum_{n=0}^\infty n^2 x^n 
	&= \sum_{n=0}^\infty {n(n + 1)(2n + 1) \over 6} x^n 
	&= \sum_{n=0}^\infty \left\{{n + 1 \choose 3} + {n + 2 \choose 3}\right\} x^n \\
{1 \over 1 - x} \sum_{n=0}^\infty n^3 x^n 
	&= \sum_{n=0}^\infty {n^2(n + 1)^2 \over 4} 
\end{align*}

Assume that we can write 
$$$n^k = c_0 {n \choose k} + c_1 {n + 1 \choose k} + \cdots + c_k {n + k \choose k}$$
then 
\begin{align*}
\sum{n=0}^\infty S_k(n) x^n 
	&= {1 \over 1 - x} \sum_{n=0}^\infty n^k x^n \\
	&= {1 \over 1 - x} \sum_{n=0}^\infty \sum_{r=0}^k c_r {n + r \choose k} x^n \\
	&= {1 \over 1 - x} \sum_{r=0}^k c_r \sum_{n=0}^\infty {n + r \choose k} x^n 
	&= 
\end{align*}

Define ${x \choose k} = {x(x-1) \cdots (x-k+1) \over k!}$ for any $x$ in
a ring where $k! \neq 0$. This naturally extetends the combinatory number
$n \choose k$.

For fixed $k$, the sum
$C_k(n) = {1 \choose k} + \cdots + {n \choose k}$ is much easier to compute than $S_k(n)$.
If we can write $S_k(n)$ as linear combination of $C_0(n), \dots, C_k(n)$, $S_k(n)$ will be
the sum of $k$-terms.

Therefore we consider to write a generic term in $S_k(n)$, that is $x^k$, as linear combination of $x \choose i$.
Assume that 
$$x^k = c_k {x \choose k} + c_{k-1}{x \choose k - 1} + \cdots + c_1 x + c_0$$
\subsection{Linear Induction Sequence}
A linear induction sequence of order $k$ is a sequence which, the since the $k$-th term, are linear combination
of the $k$-terms before it, that is 
$$a_{n+k} = c_{k-1} a_{n+k-1} + \cdots + c_0 a_n, n \geq 0$$

Let $a(x) = \sum_{n=0}^\infty a_n x^n$, 
\begin{align*}
a(x) &= \sum_{n=0}^\infty a_n x^n \\
\end{align*}


\end{document}

