\documentclass{article}
\usepackage{amsmath}
\usepackage{amssymb}
\usepackage{graphics}
\usepackage{diagrams}
\DeclareGraphicsRule{*}{mps}{*}{}

\title{Geometry \& Computer Graphics}
\author{Lgarithm}
\begin{document}
% \maketitle
% \newpage
% \tableofcontents
% \newpage

\section{Preliminary}
\subsection{Linear Space and Coordinate Space}
Let $K$ be a field, usually $\mathbb R$. 
A left $K$-linear space $L$ is an abelian group with coefficient from $K$.
Which means the elements from $L$ can be multiply by elements from $K$ on the left, 
result in an element in $L$. This multiplication is called scalar multiplication. 
Associative law and distributive law should hold for scalar multiplication. 

For a linear space, a set of minimun elemets can be chosen such that other elemets can be represent
as linear combination of the chosen elements. 
It can be proved that the number in that set is invariant if it is finite. This fix number is called 
the dimension of the linear space. And the set of ordered elements is called a basis. 
Given a basis of $n$ elements, each elements in $L$ can be represent uniquely by an $n$-tuple of coefficients.
And vice versa.

For example, let $e_1, \dots, e_n$ be a basis, and 
$$a = \lambda_1 e_1 + \cdots + \lambda_n e_n$$
Then 
$$a \quad  (\lambda_1, \dots, \lambda_n)$$
is a bijection between the linear space $L$ and the coordinate space $K^n$.


\subsection{Real Affine Space}
The real numbers $\mathbb{R}$ form a field with total order.
The linear space $\mathbb{R}^n$ is an abelian group, we also call 
it the affine space $\mathbb{A}^n(\mathbb{R})$.

\subsection{Real Projection Space}
The projection space $\mathbb{P}^n(\mathbb{R})$ is the quotient space
of $\mathbb{A}^{n+1}(\mathbb{R}) \backslash \{(0, \dots, 0)\}$ by the equavalent relation $\sim$:
\begin{align*}
\mathbb{P}^n(\mathbb{R}) &= (\mathbb{A}^{n+1}(\mathbb{R}) \backslash \{(0, \dots, 0)\}) / \sim \\
(x_1, \dots, x_{n+1}) &\sim (\lambda x_1, \dots, \lambda x_{n+1}), \lambda \neq 0
\end{align*}
$\mathbb{P}^n(\mathbb{R})$ also can be viewed as the set of all one dimensional subspaces of $\mathbb{A}^{n+1}(\mathbb{R})$.

$\mathbb{A}^n(\mathbb{R})$ can be viewed as a subset of $\mathbb{P}^n(\mathbb{R})$.


\section{The Euclid Space}
For any two points $p, q$ of the Euclid space, there is a vector $\overrightarrow{pq}$.
And for a point $p$ and a vector $\textbf{v}$, there is a point $q$ such that 
$\overrightarrow{pq} = \textbf{v}$, that is $q = p + \textbf{v}$.

In other words, the Euclid space $E^n$ is a set with the group $\mathbb{R}^n$ acts on.
More over, the action is faithful and translation.

\section{Coordinate Transform}
In a geometric point of view, 
the geometry of Eucild space is the geometry 
of point set in the space. 

To study the geometry via algebraic means, 
we assign each point in the Euclid space a tuple of numbers, 
called coordinates. 
And try to find out the properties that is independent of 
different assignments of coordiante. 

\begin{diagram}
  E^n 		& \rTo^{C_1}   & \mathbb R^n \\
  \dTo^{C_2} 	& \ruTo^{T}    & \\
  \mathbb R^n   &              &
\end{diagram}

\subsection{Two Dimensional}
Given the Euclidean plane $E^2$, each point $p$ on it can be 
represented by two real numbers, the coordinates of $p$.
\begin{align*}
  C : E^2 &\to \mathbb{R}^2 \\
  p &\to (x, y)
\end{align*}

Let $C_1, C_2$ be two coordinates systems, 
there is a coordinate transform 
\begin{equation*}
  T : \mathbb{R}^2 \to \mathbb{R}^2
\end{equation*}
such that the diagram commutates
\begin{diagram}
  E^2 		& \rTo^{C_1}	& \mathbb{R}^2 \\
  \dTo^{C_2} 	& \ruTo^{T}		& \\
  \mathbb{R}^2& &
\end{diagram}
that is $$C_1 = T \circ C_2$$

To obtain the coordinate, we must fix two axis.
\begin{figure}
\includegraphics{fig.1}
\end{figure}

\subsubsection{Translation}
Assume The coordinates of origin of $C_2$ is $(a, b)$, then 
\begin{equation*}
T(x, y) = (x - a, y - b)
\end{equation*}

\subsubsection{Rotation}
Assume the angle is $\theta$, then
\begin{equation*}
T(x, y) =
\left(\begin{matrix}
\cos \theta & \sin \theta \cr
-\sin \theta & \cos \theta
\end{matrix}\right)
\left(\begin{matrix}
x \cr y
\end{matrix}\right)
\end{equation*}
The determinant of the matrix is $1$.

\subsubsection{Reflection}
Assume the reflection axis pass through the origin of $C_1$, 
\begin{equation*}
T(x, y) =
\left(\begin{matrix}
\cos \theta & \sin \theta \cr
\sin \theta & -\cos \theta
\end{matrix}\right)
\left(\begin{matrix}
x \cr y
\end{matrix}\right)
\end{equation*}
The determinant of the matrix if $-1$.

% \subsubsection{Composition}


\subsection{Three Dimensional}


\section{Object Transform}
% Given a coordinate on a Euclid space, 
An object transform is a transform on the Euclid space
$$O : E^n \to E^n$$
Given a coordinate $C : E^n \to \mathbb R^n$
we have the following commutative diagram:
\begin{diagram}
  E^n & \rTo^{O} & E^n \\
  \dTo<{C} & & \dTo>{C} \\
  \mathbb R^n & \rTo_{T} & \mathbb R^n
\end{diagram}
\begin{align*}
  T \circ C &= C \circ O \\
  T &= C \circ O \circ C^{-1}
\end{align*}

\subsection{Two Dimension}
\begin{align*}
T(x, y) &= (px + qy + a, rx + sy + b) \\
\det \left(\begin{matrix}
p & q \cr r & s
\end{matrix}\right) & \neq 0
\end{align*}

\section{Projection}
\subsection{Three Dimension to To Dimension projection}
Given a point $p = (x, y, z)$ and a plane $S$ in $\mathbb{R}^3$.

\end{document}
