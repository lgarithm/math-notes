\documentclass{article}
\usepackage{amsmath}
\usepackage{amssymb}

\newtheorem{prop}{Proposition}

\def\abs#1{\left| {#1} \right|}
\def\ceil#1{\lceil {#1} \rceil}
\def\zmod#1{\mathbb Z / {#1} \mathbb Z}
\def\zmodu#1{U(\mathbb Z / {#1} \mathbb Z)}
\def\ann{\mathop{\mathrm{ann}}}
\def\ord{\mathop{\mathrm{ord}}}

\title{Discrete Root in $\mathbb Z / n \mathbb Z$}

\author{Lgarithm\\\mbox{lgarithm@gmail.com}}


\begin{document}

\maketitle
\begin{abstract}
Given integer $a, m$ and $n$, it's easy to compute the exponenet $a^m$ modulo $n$,
but the discrete root and discrete logarithm are not as easy as exponent.

In this article, we will study the equation
$$x^m \equiv b \pmod n$$
in detail and give algorithms to determine the existence of solution,
and number of solutions if exists.

Note that when $m = 2$, this is exactly the quadratic residue problem.
\end{abstract}

\section*{Notations}
\begin{tabular}{l l}
$[a]_n$    & the equivalent class of $a$ modulo $n$ \\
$v_p(a)$   & index of prime $p$ in a ration number $a$ \\
$R^\ast$   & $R \backslash \{0\}$ \\
$U(R)$     & units of a ring $R$ \\
\end{tabular}
\tableofcontents

\newpage

\section{Introduction}
Let $n = \prod p_i^{e_i}$ be the factorization of $n$, and $q_i = p_i^{e_i}$.
According to the CRT, we only need to solve $x_i^m \equiv b \pmod {q_i}$ for each $q_i$.

For $q = p^e$, there are three situations,
\begin{enumerate}
\item $(b, p) = 1$
\item $p \mid b$ and $q \not| b$
\item $b \equiv 0 \pmod q$
\end{enumerate}
In terms of the ring $\zmod q$, those cases corresponding to $b$ is a unit, $b$ is a zero factor
and $b$ is 0.

For the first case, $b \in \zmodu q$, and $\zmodu q$ is a cyclic
group of order $\phi(q)$ when $p \neq 2$, for $q$ has primitive roots when $p$ is odd.

For the second case, notice that $b$ is contained in the unique maxical ideal of $\zmod q$,
so we may consider $\zmod q$ as a discrete valuation ring though it's not a domain.

For the thired case, which indeed is the simplest one. We have $m v_p(x) \geq e$
i.e. $v_p(x) \geq \ceil{e \over m}$.

\section{Modular Arithmetic}
This section we discuss the linear equation $ax + by = c$ where $a, b, c$ are integers.
The results are indeed elementary, we just presented here for sake of completeness.

\subsection{Divisability and Congruence}
The following three statements are equivalent according to definition.
\begin{enumerate}
\item $a \equiv b \pmod n$
\item $n \mid a - b$
\item $\exists k \in \mathbb Z$, such that $kn = a - b$
\end{enumerate}

We will often switch among any of these three equivalent representations.

\subsection{Modular Inverse}
For integer $n \neq 0$ and $a$, we say $b$ is a modular inverse of $a$ modulo $n$ if $ab \equiv 1 \pmod n$.
Let $A = \{a_1, \dots, a_{\phi(n)}\}$ be a reduced residue system modulo $n$,
for $a \in A$, we have $aa_1, \dots, aa_{\phi(n)}$ is a also a reduced residue system modulo $n$.
Therefore, there must be some $a_i$ such that $aa_i \equiv 1 \pmod n$.


\subsection{The Primary Ideal $L(a, b)$}
For non-zero integers $a, b$, define $L(a, b) = \{ax + by \mid x, y \in \mathbb Z\}$.
When $(a, b) = 1$, take $x_0 \equiv a^{-1} \pmod b$, then $ax_1 \equiv 1 \pmod b$,
equivalently $b \mid ax_1 - 1$, that is to say, exists $y_1 \in \mathbb Z$, suth that $ax_1 + by_1 = 1$.
Therefore, for $k \in \mathbb Z$, take $x = kx_1, y = ky_1$, we got $k \in L(a, b)$,
$L(a, b) = \mathbb Z$ when $(a, b) = 1$.

When $(a, b) = d > 1$, it's easy to see that $L(a, b) = \{dk \mid k \in \mathbb Z\}$.

\subsection{The Equation $ax + by = c$}
$ax + by = c$ has a solution if and only if $c \in L(a, b)$, equivalently $(a, b) \mid c$.

\subsection{The Equation $ax \equiv b \pmod n$}
We shall determine all solutions for the equation of $x$ modulo $n$.
When $(a, n) = 1$, there is only one solution $x \equiv ba^{-1} \pmod n$ module $n$.
All solutions are $ba^{-1} + tn, t \in \mathbb Z$.


When $(a, n) = d > 1$, rewrite the equation in the form
$ax + ny = b$, when $(a, n) \not| b$, it has no solutions.
otherwise let $a = a_0 d$, $b = b_0d$, $n = n_0 d$, where $d = (a, n)$.
then the equation becomes $a_0 x + n_0 y = b_0$, $(a_0, n_0) = 1$,
equivalently $a_0 x \equiv b_0 \pmod n_0$, $(a_0, n_0) = 1$.
We already know all solutions for this equation are
$b_0 a_0^{-1} + t n_0, t \in \mathbb Z$, where $a_0^{-1}$ is the modular inverse of
$a_0$ modulo $n_0$. Since $n_0 = {n \over d}$, we have $d$ solutions modulo $n$,
namely $b_0 a_0^{-1} + k n_0, k = 0, \dots, d - 1$.


\section{Cyclic group}
In this section we determine the number of solutions of $x^m = b$ in a finite cyclic group $G$.
Let $g$ be a generator of $G$, we have the canonical isomorphism
$$\sigma_g: G \to \zmod n, \quad \sigma(g^k) = [k]_n$$ where $n = \abs G$.

Then the equation $x^m = b$ in $G$ becomes the familiar one
$$m \sigma_g(x) \equiv \sigma_g(b) \pmod n$$
it has $(m, n)$ solutions when $(n, m) \mid \sigma_g(b)$
and has none otherwise. Notice that the number of solutions is independent of choice of $g$.

The function $\sigma_g$ is well defined, but for $x \in G$, it's usually not easy to find a
$k \in \mathbb Z$ such that $x^k = g$. And find a generator $g$ in $G$ is not always easy either.
However, we can determine the existence of solutions of $x^m = b$ without $g$ and $\sigma_g$.

Note that $\sigma_g(x)$ is actually the discrete logarithm of $x$ based $g$.

\subsection{The Order of an Element}
For any group $G$ and $b \in G$, if there exists $r \in \mathbb Z^\ast$ such that $b^r = e_G$,
we say $b$ is an element of finite order. If $b$ is an element of finite order,
there must be a minimum positive integer $\ord_G(b)$, we call the order of $b$.
In fact, $\ann_G(b) = \{r \mid b^r = e_G\}$ forms an ideal of $\mathbb Z$,
$\ord_G(b)$ is the generator of $\ann_G(b)$. $\abs G \in \ann_G(b)$ when $\abs G$ is finite.

Now we give an algorithm to determine $\ord_G(b)$ for finite group $G$.
The algorithm is based on the following fact.

If $b^r = e$, $p \mid r$, and $b^{r \over p} \neq e$, then $v_p(r) = v_p(\ord_G(b))$.

When we take all prime factors of $r$, we got a predicate to decide if $r$ equals $\ord_G(b)$.
If $b^r = e$, and $b^{r \over p} \neq e$ for all prime factors of $r$, then $r = \ord_G(b)$.

Now when we got an element $r$ of $\ann_G(b)$, and the factorization of $r$,
we can find out $\ord_G(b)$.

In case when $\abs G$ is finite, $\abs G \in \ann_G(b)$.

\subsection{Invariant of $\sigma_g$}
Given group $G$, $\ord_G : G \to \mathbb Z$ is only determined by $G$.
$\sigma_g$ also should have some invariant which is independent of $g$.
Let $G \cong \zmod n$ without losing generality.
Assume $\sigma_g(x) = k$, i.e. $kg \equiv x \pmod n$, or $k \equiv xg^{-1} \pmod n$
Note that $(g, n) = 1$, therefore $(x, n) = (k, n) = (\sigma_g(x), n)$, which is
only depent on $x$.

\section{Discrete Valuation Ring}
This section we discuess the properties of $\mathbb Z / q \mathbb Z$ as a DVR.
The DVR studied in this seciton is not necessarily a domain.

We always assuming the commutative ring has a multiplicative identity.
A Noetherian ring is a commutative ring such that all ideals are finite generated.

\subsection{Local Ring}
A Noetherian ring is local if it has unique maximal ideal.
The unique maximal ideal is the set of all non-unit of $R$.
The ring $\mathbb Z / q \mathbb Z$ is a local ring, whose maximal ideal is
$\{[a]_q \mid p \vert a\}$

\subsection{Discrete Valuation Ring}
A ring is call a discrete valuation ring(DVR) if it is local, and the maximal ideal is primary.
The ring $\mathbb Z / q \mathbb Z$ is a DVR.

Let $g$ be one generator of the unique maximal ideal.
If DVR $R$ is a domain, every non-zero elements $a$ of $R$ can be represent by a pair of $(n, u)$ as $a = g^n u$
where $n \in \mathbb N$, $u$ is a unit of $R$.
The natural number $n$ is called the order of $a$, and it's independent of the choice of $g$.

However $\zmod q$ is not a domain, but it has samiliar property.

\section{Structure of $\zmodu{2^m}$}
$\zmodu{2^m}$ is not a cyclic group since $2^m$ has no primitive root when $m > 2$.
But it is almost a cyclic group, in concrete words, we have
$$\zmodu{2^m} \cong \zmod 2 \oplus \zmod{2^{m-2}}$$
and $5$ is of order $2^{m-2}$.

Remember that the order of $5$ module $2^m$ must be a divisor of $\varphi(2^m) = 2^{m-1}$.
We prove it equals $2^{m-2}$ by showing that $5^{2^{m-3}} \equiv 2^{m-1} + 1 \pmod{2^m}$.

Assume $5^{2^{k-3}} \equiv 2^{k-1} + 1 \pmod{2^k}$, then $5^{2^{k-3}} = 2^k t + 2^{k-1} + 1$ for some integer $t$.
Then $5^{2^{k-2}} = (2^k t + 2^{k-1} + 1)^2 = 2^{k+1}(2^{k-1}t^2 + 2^{k-1}t + 2^{k-3} + t) + 2^k + 1$,
therefore $5^{2^{k-1}} \equiv 2^k + 1 \pmod{2^{k + 1}}$.

Notice that $5$ generates the elements of $4k + 1$,
and $-1 \equiv 2^m - 1 \pmod{2^m}$ is an element of order $2$, also of form $4k + 3$.

It's not hard to prove that for every $a \in \mathbb Z$, there is a unique pair of $(l, r)$,
$0 \leq l < 2^{m-2}$, $0 \leq r < 2$ such that $5^l(-1)^r \equiv a \pmod{2^m}$.

\section{Structure of $\zmod{2^m}$}
$\zmod{2^m}$ is also a DVR.

\end{document}
